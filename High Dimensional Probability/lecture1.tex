\documentclass[../main]{subfiles}

\begin{document}

\section{Big Data}
There are two notions of big data: \begin{itemize}
  \item The number of observations (data points) is big. e.g. the income of 3,000,000 residents. However, the dimension of the data is $1$.
  \item The number of dimensions (parameters) is big. e.g. Images of people have a lot of pixels and each pixel is a parameter.
  \item \item Text, sound, video, genome, medical history are all high dimensional observations.
\end{itemize}

\begin{itemize}
  \item The number of observations is usually not a problem but the number of dimensions is usually more problematic. Empiricaly, it is exponentially harder to deal with large number of dimensions.
  Statistics tells us that more data points only help. Why?
  \item To demonstrate the above point, consider Monte-Carlo Integration. 
  \[
    \int_{0}^1\dots \int_0^1 f(x_1,\dots,x_d)dx_1\dots x_d = \int_{(0,1)^d}f(x)dx
  \]
\end{itemize}

\end{document}
