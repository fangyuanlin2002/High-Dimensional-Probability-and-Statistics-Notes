\documentclass[../main]{subfiles}
\title{Common Derivatives}
\author{Fangyuan Lin\\
\hyperref[website]{https://fangyuanlin2002.github.io/}}
\date{}
\begin{document}
\maketitle
\tableofcontents
\section{Remarks}
\begin{itemize}
    \item Find it annoying to always forget the common multivariate derivatives? Me too.
    \item All variables are assumed to be vectors of appropriate dimensions and $\vec x$ is written as $x$ for convenience. 
\end{itemize}
\section{Common Derivatives}

\subsection{Linear Form}
\begin{bbox}{Linear Form}
    Let 
    \[
    f(x) = a^T x.
    \]
    Then
    \[
    \nabla a^T x = a
    \]

    \begin{proof}
        \begin{align*}
            f(x) &= \sum_i a_i x_i\\
            \frac{\partial f}{\partial x_i}&=a_i
        \end{align*}
    \end{proof}
\end{bbox}

\subsection{Affine Function}
\begin{bbox}{Affine Function}
    Let 
    \[
    f(x) = Ax + b
    \]
    Then 
    \[
    J_f(x) = A
    \]
    \begin{proof}
        \begin{align*}
            f_i(x) &= \sum_jA_{ij}x_j + b_i\\
            \frac{\partial f_i}{\partial x_j} &= A_{ij}
        \end{align*}
    \end{proof}
\end{bbox}
\subsection{Bilinear Form / Inner Product}

\begin{bbox}{Bilinear Form}
    Let 
    \[
    f(x, y) = x^T A y
    \]
    Then 
    \[
    \nabla_x f = Ay, \quad \nabla_y f = A^T x
    \]
    \begin{proof}
        Fix y. Then 
        \begin{align*}
            f(x) &= \sum_i x_i (Ay)_i\\
            \frac{\partial f}{\partial x_i} &= (Ay)_i
        \end{align*}
        Fix x. Then 
        \begin{align*}
            f(y) &= \sum_i (x^T A)_i y_i\\
            \frac{\partial f}{\partial y_i} &= (x^T A)_i\\
            \nabla_y f&= (x^TA)^T = A^T x
        \end{align*}
    \end{proof}
\end{bbox}
\subsection{Quadratic Form}
\begin{bbox}{Quadratic Form}
     Let 
    \[
    f(x) = x^T A x
    \]
    Then 
    \[
    \nabla f = (A+A^T)x
    \]
    \begin{align*}
            f(x) &= \sum_i x_i (Ax)_i\\
            &= \sum_i x_i (\sum_{j}A_{ij}x_j)\\
            &= \sum_i\sum_{j}x_{i}A_{ij}x_j\\
            \frac{\partial f}{\partial x_k} &= \sum_{j\neq k}A_{kj}x_j + 2A_{kk}x_{k} + \sum_{i\neq k}A_{ik}x_{i}\\
            &= (Ax)_k + (A^Tx)_k
        \end{align*}
\end{bbox}

\subsection{Squared Norm}
\begin{bbox}{Squared Norm}
    Let 
    \[
    f(x) = \|x\|^2 = x^T x.
    \]
    Then
    \[
    \nabla f (x) = 2x
    \]
    \begin{proof}
        Note that his is a special case of quadratic form with $A = I$. Also one can consider:
        \begin{align*}
        f(x) &= \sum_{i}x_i^2\\
        \frac{\partial f}{\partial x_i} &= 2x_i
    \end{align*}
    \end{proof}
    
\end{bbox}
\end{document}